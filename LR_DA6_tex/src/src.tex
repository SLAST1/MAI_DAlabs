\section{Описание}
Требуется реализовать класс для хранения «длинных» чисел, реализовать операции
сложения, вычитания, умножения, деления и возведения в степень. Сложение выполняется очень просто, складываем разряды, в текущий разряд записываем остаток,
а целую часть от деления запоминаем для следующего разряда. Вычитание выполняется аналогичным образом, с запоминанием, требуется ли нам «занять» число.
Умножение выполняется аналогичным образом, только для каждой пары разрядов
и последующих сдвигов для больших разрядов. Деление в степень выполняется начиная с больших разрядов, берется минимальное число, которое делится нацело,
при помощи бинарного поиска и вычитается как в обычном делении стобком, и так
до конца разрядов. Возведение в степень выполняется, используя уже написанное
умножение, рекурсивно, если степень четная, то умножаем полученный результат из
рекурсии сам на себя и делим степень пополам, если степень нечетная, то умножаем на исходное число и вычитаем из степени 1.Сравнения происходят очень просто,
сначала сравниваем длины, если равны, то сравниваем по разрядам.

В программе bigiht.cpp содержатся основные конструкторы и функции для реализации нашей программы. А так же операторы вычислительных действий (сложения, вычитания, умножения, деления, сравнения)

\pagebreak

\section{Исходный код}

Код: main.cpp
\begin{lstlisting}[language=C++]
#include <iostream>
#include <iomanip>
#include <string>
#include <vector>
#include <cmath>
#include "bigint.hpp"

int main()
{
NBigInt::TBigInt num1, num2;
NBigInt::TBigInt zero("0");
char action = '?';
while(std::cin >> num1 >> num2 >> action)
{
if(action == '+')
{
NBigInt::TBigInt res = num1 + num2;
std::cout << res << "\n";
}
else if(action == '-')
{
if(num1 < num2)
{
std::cout << "Error\n";
continue;
}
NBigInt::TBigInt res = num1 - num2;
std::cout << res << "\n";
}
else if(action == '*')
{
NBigInt::TBigInt res = num1 * num2;
std::cout << res << "\n";
}
else if(action == '^')
{
if(num1 == zero && num2 == zero)
{
std::cout << "Error\n";
continue;
}
NBigInt::TBigInt res = num1 ^ num2;
std::cout << res << "\n";
}
else if(action == '/')
{
if(num2 == zero)
{
std::cout << "Error\n";
continue;
}
NBigInt::TBigInt res = num1 / num2;
std::cout << res << "\n";
}
else if(action == '<')
{
if(num1 < num2) { std::cout << "true\n"; }
else { std::cout << "false\n"; }
}
else if(action == '>')
{
if(num1 > num2) { std::cout << "true\n"; }
else { std::cout << "false\n"; }
}
else if(action == '=')
{
if(num1 == num2) { std::cout << "true\n"; }
else { std::cout << "false\n"; }
}
else { std::cout << "Error\n"; }
}
return 0;
}
\end{lstlisting}

\pagebreak

Код: digiht.cpp
\begin{lstlisting}[language=C++]
#include <iostream>
#include <iomanip>
#include <vector>
#include <string>
#include <algorithm>
#include "bigint.hpp"

namespace NBigInt
{

TBigInt::TBigInt(std::string const& str) { this->Init(str); }

void TBigInt::Init(std::string const& str)
{
data.clear();
int size = static_cast<int>(str.size());
for (int i = size - 1; i >= 0; i = i - TBigInt::RADIX)
{
if (i < TBigInt::RADIX) { data.push_back(static_cast<int32_t>(atoll(str.substr(0, i + 1).c_str()))); }
else { data.push_back(static_cast<int32_t>(atoll(str.substr(i - TBigInt::RADIX + 1, TBigInt::RADIX).c_str())));}
}
DeleteLeadingZeros();
}

void TBigInt::DeleteLeadingZeros()
{
while(!data.empty() && data.back() == 0) { data.pop_back(); }
}

std::istream& operator>>(std::istream& in, TBigInt& rhs)
{
std::string str;
in >> str;
rhs.Init(str);
return in;
}

std::ostream& operator<<(std::ostream& out, TBigInt const& rhs)
{
if(rhs.data.empty())
{
out << "0";
return out;
}
out << rhs.data.back();
for(int i = rhs.data.size() - 2; i >= 0; --i)
{
out << std::setfill('0') << std::setw(TBigInt::RADIX) << rhs.data[i];
}
return out;
}

TBigInt const operator+(TBigInt const& lhs, TBigInt const& rhs)
{
TBigInt res;
int32_t carry = 0;
size_t n = std::max(lhs.data.size(), rhs.data.size());
res.data.resize(n);
for(size_t i = 0; i < n; ++i)
{
int32_t sum = carry;
if(i < rhs.data.size()) { sum += rhs.data[i]; }
if(i < lhs.data.size()) { sum += lhs.data[i]; }
carry = sum / TBigInt::BASE;
res.data[i] = sum % TBigInt::BASE;
}
if(carry != 0) { res.data.push_back(static_cast<int32_t>(1)); }
res.DeleteLeadingZeros();
return res;
}

TBigInt const operator-(TBigInt const& lhs, TBigInt const& rhs)
{
TBigInt res;
int32_t carry = 0;
size_t n = std::max(lhs.data.size(), rhs.data.size());
res.data.resize(n);
for(size_t i = 0; i < n; ++i)
{
int32_t sub = lhs.data[i] - carry;
if(i < rhs.data.size()) { sub -= rhs.data[i]; }
carry = 0;
if(sub < 0)
{
carry = 1;
sub += TBigInt::BASE;
}
res.data[i] = sub % TBigInt::BASE;
}
res.DeleteLeadingZeros();
return res;
}

TBigInt const operator*(TBigInt const& lhs, TBigInt const& rhs)
{
if(rhs.data.size() == 1) { return lhs.MultShort(rhs); }
if(lhs.data.size() == 1) { return rhs.MultShort(lhs); }
TBigInt res;
size_t n = lhs.data.size() * rhs.data.size();
res.data.resize(n + 1);
int32_t k = 0;
int32_t r = 0;
for(size_t i = 0; i < lhs.data.size(); ++i)
{
for(size_t j = 0; j < rhs.data.size(); ++j)
{
k = rhs.data[j] * lhs.data[i] + res.data[i+j];
r = k / TBigInt::BASE;
res.data[i+j+1] = res.data[i+j+1] + r;
res.data[i+j] = k % TBigInt::BASE;
}
}
res.DeleteLeadingZeros();
return res;
}

TBigInt const TBigInt::MultShort(TBigInt const& rhs) const
{
int32_t carry = 0;
int32_t mult = 0;
size_t size = data.size();
TBigInt res;

for (size_t i = 0; i < size; ++i)
{
mult = data[i] * rhs.data[0] + carry;
res.data.push_back(mult % BASE);
carry = mult / BASE;
}
if (carry != 0) { res.data.push_back(carry); }
res.DeleteLeadingZeros();
return res;
}

TBigInt const operator^(TBigInt const& lhs, TBigInt const& power)
{
TBigInt res("1");
TBigInt two("2");
TBigInt one("1");
TBigInt zero("0");
if(power == zero) { return res; }
if(power == one || lhs == one) { return lhs; }
if(power.data[0] % 2 == 0)
{
TBigInt res = lhs ^ (power / two);
return res * res;
}
else
{
TBigInt res = lhs ^ (power - one);
return lhs * res;
}
}

void TBigInt::ShiftRight()
{
if (data.size() == 0)
{
data.push_back(0);
return;
}
data.push_back(data[data.size() - 1]);
for (size_t i = data.size() - 2; i > 0; --i) { data[i] = data[i - 1]; }
data[0] = 0;
}

TBigInt const operator/(TBigInt const& lhs, TBigInt const& rhs)
{
TBigInt curr, res;
size_t lhs_size = lhs.data.size();
res.data.resize(lhs_size);
int l = 0;
int r = TBigInt::BASE;
int m = 0;
int data_res = 0;
for(int i = lhs_size - 1; i >= 0; --i)
{
m = 0;
l = 0;
r = TBigInt::BASE;
curr.ShiftRight();
curr.data[0] = lhs.data[i];
curr.DeleteLeadingZeros();
while(l <= r)
{
m = (l + r) / 2;
if(rhs * TBigInt(std::to_string(m)) <= curr)
{
data_res = m;
l = m + 1;
}
else { r = m - 1; }
}
res.data[i] = data_res;
curr = curr - rhs * TBigInt(std::to_string(data_res));
}
res.DeleteLeadingZeros();
return res;
}

bool operator<(TBigInt const& lhs, TBigInt const& rhs)
{
if(lhs.data.size() != rhs.data.size()) { return lhs.data.size() < rhs.data.size(); }
for(int i = lhs.data.size() - 1; i >= 0; --i)
{
if(lhs.data[i] != rhs.data[i]) { return  lhs.data[i] < rhs.data[i]; }
}
return false;
}

bool operator==(TBigInt const& lhs, TBigInt const& rhs) { return !(lhs < rhs) && !(rhs < lhs); }
bool operator!=(TBigInt const& lhs, TBigInt const& rhs) { return !(lhs == rhs); }
bool operator<=(TBigInt const& lhs, TBigInt const& rhs) { return !(rhs < lhs); }
bool operator>(TBigInt const& lhs, TBigInt const& rhs) { return !(lhs <= rhs); }
bool operator>=(TBigInt const& lhs, TBigInt const& rhs) { return !(lhs < rhs); }

}
\end{lstlisting}

\section{Консоль}

\begin{alltt}
(py37) ~ /DA_labs/lab6$ make
g++ -g -O2 -pedantic -std=c++17 -Wall -Wextra -Werror main.cpp -o solution
(py37) ~ /DA_labs/lab6$ cat test_sum.txt
050604130001
021410140107
+
(py37) ~ /DA_labs/lab6$ ./solution <test_sum.txt
72014270108
(py37) ~ /DA_labs/lab6$ cat test_sub.txt
040500130012
021410140107
-
(py37) ~ /DA_labs/lab6$ ./solution <test_sub.txt
19089989905
(py37) ~ /DA_labs/lab6$ cat test__mult.txt
050604130001
021410140107
*
(py37) ~ /DA_labs/lab6$ ./solution <test_mult.txt
1083441513314252050107
(py37) ~ /DA_labs/lab6$ cat test_power.txt
14
02
^
(py37) ~ /DA_labs/lab6$ ./solution <test_power.txt
196
(py37) ~ /DA_labs/lab6$ cat test_div.txt
050604130001
031014
/
(py37) ~ /DA_labs/lab6$ ./solution <test_div.txt
1631654
(py37) ~ /DA_labs/lab6$ cat test_less.txt
001203
100103
<
(py37) ~ /DA_labs/lab6$./solution <test_less.txt
true
(py37) ~ /DA_labs/lab6$ cat test_more.txt
9927
0234
>
(py37) ~ /DA_labs/lab6$ ./solution <test_more.txt
true
(py37) ~ /DA_labs/lab6$ cat test_eq.txt
9927
00927
=
(py37) ~ /DA_labs/lab6$ ./solution <test_eq.txt
true
\end{alltt}

\pagebreak

