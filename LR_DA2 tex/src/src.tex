\section{Описание}
Процесс разработки можно разделить на два этапа: создание и отладка структуры красно-черное дерево и реализация парсера команд (словаря), работающего с этим деревом.

Дерево должно хранить в своих вершинах слова (в формате строки) и соответствующие им числа. Так как слова в словаре должны быть отсортированы в алфавитном
порядке, ключом будет считаться именно слово, а соответствующее ему число – значением.

Красно-черное дерево – самобалансирующееся дерево, узлы которого имеют атрибут
цвета. При этом, дерево должно удовлетворять условиям: 1. Узел может быть либо
красным, либо чёрным и имеет двух потомков; 2. Корень – чёрный. 3. Все листья
— чёрные. 4. Оба потомка каждого красного узла — чёрные. 5. Любой простой путь
от узла-предка до листового узла-потомка содержит одинаковое число чёрных узлов. Для соблюдения этих условий после каждой операции вставки/удаления узла
должна проводиться перебалансировка. Время, затрачиваемое на доступ к элементам, составляет $O(log(n))$, где $n$ – число элементов в нем.

Перебаланисировка производится с помощью операций поворота и перекрашивания
узлов, которые применяются в зависимости от текущей расцветки и положений узлов дерева. В ходе перебалансировки количество черных узлов на пути из элемента,
для которого операция перебалансировки была вызвана до всех его листовых потомков уравниваются, однако так как при этом их суммарное число могло измениться,
в некоторых случаях операция рекурсивно вызывается для его потомков, пока не
отбалансируется все дерево начиная с корня.

В ходе разработки структуры «красно-черное дерево» также необходимо реализовать
операции сохранения его в файл и загрузки из файла. Элементы дерева записываются в файл по очереди их прохождения в процессе обхода в глубину, при этом,
вначале записывается текущий элемент, затем – оба его потомка, начиная с левого.
Таким образом, во время чтения дерева из заранее сохраненного файла при обнаружении очередного элемента мы можем быть уверены, что следующий за ним элемент
является его левым потомком. Если был найден лист (элемент в котором лежит строка длины ноль), то мы понимаем, что текущая ветвь закончилась, и далее записан
правый потомок родителя считанного ранее узла.

Для того чтобы не тратить время на попытки расшифровки файлов, которые были
созданы не программой словаря, в начало файла добавляется маркер, состоящий из
трех символов. При чтении дерева файла структура строится отдельно от текущего
дерева, и только если чтение прошло без ошибок, текущее дерево заменяется считанным

В самом словаре производится непрерывная обработка входных строк, в соответствии с результатом этой обработки для каждой введенной команды вызывается
метод заранее созданного дерева. Методы в большинстве своем возвращают код завершения, в соответствии с которым осуществляется вывод на экран информации о
результате выполнения команды.

\pagebreak

\section{Исходный код}
Вначале создаем класс узла дерева. В нем хранится ключ (указатель на строку),
значение (переменная типа unsigned long long) и цвет (переменная типа char). Также
в узле хранятся указатели на родителя и обоих детей этого узла.

Для компактности отчета здесь и далее я буду приводить методы класса только в таблице

\begin{lstlisting}[language=C++]
const char BLACK = 1;
const char RED = 2;

class TRBNode {
public: 
char* Key; 
unsigned long long Value;
char Colour = BLACK;

TRBNode* Parent = nullptr;
TRBNode* Left = nullptr;
TRBNode* Right = nullptr;
\end{lstlisting}

Также реализуем вспомогательную функцию strequal, работающую аналогично strcmp,
однако не реагирующую на различный регистр букв сравниваемых строк.

Реализуем класс дерева.
\begin{lstlisting}[language=C++]
class TRBTree {
private:
	TRBNode* TreeRoot = nullptr;
}
\end{lstlisting}

В функции main будем производить обработку команд, предварительно создав дерево, в которое будем записывать слова.

\begin{lstlisting}[language=C++]
int main() {
TRBTree maintree; 

char *inpstr = (char*)calloc(260,sizeof(char)); 
char *word = (char*)calloc(260,sizeof(char)); 
unsigned long long inpval;

if(inpstr == NULL || inpstr == nullptr || word == NULL || word == nullptr) {
printf("ERROR: allocation error\n");
return -1;
}

while (scanf("%s", inpstr) != EOF) {
if(strlen(inpstr)>256) {
printf("ERROR: uncorrect input\n");
continue;
}
if (inpstr[0] == '+') {
if(scanf("%s %llu", word, &inpval) == EOF) {
break;
}
if(strlen(word)>256) {
printf("ERROR: uncorrect input\n");
continue;
}
int mrk = maintree.Add(word, inpval);
if (mrk == 0) {
printf("OK\n");
}
else {
if (mrk == -7) {
printf("Exist\n");
}
else
if (mrk == -3) {
printf("ERROR: empty input\n");
}
else
if (mrk == -4) {
printf("ERROR: untraced allocation error\n");
}
else
if (mrk == -1) {
printf("ERROR: out of memory\n");
}
else {
printf("ERROR: unknown error\n");
}
}
}
else 
if (inpstr[0] == '-') {
if(scanf("%s", word) == EOF) {
break;
}
if(strlen(word)>256) {
printf("ERROR: uncorrect input\n");
continue;
}
int mrk = maintree.Remove(word);
if (mrk == 0) {
printf("OK\n");
}
else 
if (mrk == -8) {
printf("NoSuchWord\n");
}
else
if (mrk == -1) {
printf("ERROR: out of memory\n");
}
else {
printf("ERROR: unknown error\n");
}
}
else
if (inpstr[0] == '!') {
std::string path;
if(scanf("%s", word) == EOF) {
break;
}
if (strcmp(word, "Save") == 0) {
std::cin >> path;

int mrk = maintree.SaveToDisk(path); 
if (mrk == 0) {
printf("OK\n");
continue;
}
else 
if (mrk == 1) {
printf("OK\n");
continue;
}
else 
if (mrk == -1) {
printf("ERROR: unable to open file\n");
continue;
}
else 
if (mrk == -2) {
printf("ERROR: unable to write file\n");
continue;
}
else 
if (mrk == -3) {
printf("ERROR: file acsess error\n");
continue;
}
else 
{
printf("ERROR: something gone wrong\n");
continue;
}
}
else
if (strcmp(word, "Load") == 0) {
std::cin >> path;

int mrk = maintree.LoadFromDisk(path); 
if (mrk == 0) {
printf("OK\n");
continue;
}
else 
if (mrk == -1) {
printf("ERROR: file is damaged\n");
continue;
}
else 
if (mrk == -2) {
printf("ERROR: wrong format of file\n");
continue;
}
else 
if (mrk == -3) {
printf("ERROR: file acsess error\n");
continue;
}
else 
{
printf("ERROR: something gone wrong\n");
continue;
}
}

}
else {
TRBNode* res = maintree.Find(inpstr);
if (res != nullptr) {
printf("OK: %llu\n",res->Value);
}
else {
printf("NoSuchWord\n");
}
}
}

maintree.Destroy(); 
free(inpstr);
free(word);

return 0;
}
\end{lstlisting}

\pagebreak

\section{Консоль}

\begin{alltt}
(py37) ~ /DA/lab2$ g++ -g -Wall -o lab2ex Prog/DA_lab_2_nostl.cpp
(py37) ~ /DA/lab2$ cat tests/test1
+ a 1
+ A 2
+ aa 18446744073709551615
aa
A
-A
a
(py37) ~ /DA/lab2$ ./lab2ex <tests/test1
OK
Exist
OK
OK: 18446744073709551615
OK: 1
OK
NoSuchWord
\end{alltt}

\pagebreak

\begin{alltt}
(py37) ~ /DA/lab2$ cat tests/test2
! Save empty.b
! Load asdfghj
+ a 11
a
b
-a
! Save file
a
+ b 7
b
! Load file
a
b
(py37) ~ /DA/lab2$ ./lab2ex <tests/test2
OK
ERROR: file acsess error
OK
OK: 11
NoSuchWord
OK
OK
NoSuchWord
OK
OK: 7
OK
NoSuchWord
NoSuchWord
\end{alltt}

\pagebreak
