\CWHeader{Лабораторная работа \textnumero 7}

\CWProblem{
При помощи метода динамического программирования разработать алгоритм решения задачи, определяемой своим вариантом; оценить время выполнения
алгоритма и объем затрачиваемой оперативной памяти. Перед выполнением задания
необходимо обосновать применимость метода динамического программирования.

Разработать программу на языке C или C++, реализующую построенный алгоритм.

Формат входных данных:
В первой строке заданы $1 \leq n \leq 100 и 1 \leq m \leq 5000$. В последующих n строках через пробел заданы параметры предеметов: $w_i$ и $c_i.$


{\bfseries Вариант:} У вас есть рюкзак, вместимостью $m$, а так же $n$ предметов, у каждого из которых есть вес $w_i$ и стоимость $c_i$.
Необходимо выбрать такое подмножество $I$ из них, чтобы:
$\sum_{i \in I} w_i \leq m$ и $(\sum_{i \in I} c_i) \times | I |$ является максимальной из всех возможных.
$| I |$ - мощность множества $I$.

}
\pagebreak

