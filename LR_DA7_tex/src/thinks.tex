\section{Выводы}

В ходе выполнения лабораторной работы я изучил классические задачи динамического программирования и их методы решения, реализовал алгоритм для своего
варианта задания.

Также я познакомился с $std::bitset$ для уменьшении потребляемой программой памяти, узнал, что $std::vector$ имеет спецификацию $std::vector<bool>$, которая тоже эффективна по памяти, как и $std::bitset$.

Динамическое программирование позволяет разработать точные и относительно быстрые алгоритмы для решения сложных задач, в то время, как переборное решение
слишком медленное, а жадный алгоритм не всегда даёт правильный результат.

Например, известную NP-полную задачу о коммивояжере [3] можно решить с помощью динамического программирования по подмножествам за $O(n^2 \times 2^n)$, что гораздо
быстрее перебора за $O(n!)$.

\pagebreak
