\section{Тест производительности}

Тест состоит из ввода графа и запуска алгоритма Джонсона 50000, 100000 и 200000 раз.

Моя реализация:
\begin{alltt}
(py37) ~ /DA_labs/lab7$ make
g++ -g -O2 -pedantic -std=c++17 -Wall -Wextra -Werror main.cpp -o solution
(py37) ~ /DA_labs/lab8$ make bench
g++ -g -O2 -pedantic -std=c++17 -Wall -Wextra -Werror benchmark.cpp -o benchmark
(py37) ~ /DA_labs/lab8$ ./benchmark
Enter graph:
5 4
1 2 -1
2 3 2
1 4 -5
3 1 1
Enter test count:
50000
Time for Johnson on 50000 tests: 3 seconds

\end{alltt}

Алгоритм работает за $O(n * m + n^2 * lnn)$, в нашем случае $n = m -> O(n^2 * lnn)$, что лучше чем алгоритм Флойда-Уоршела, имеющего сложность $O(n^3)$ и явно лучше наивного перебора за $O(n^4)$.
\pagebreak

