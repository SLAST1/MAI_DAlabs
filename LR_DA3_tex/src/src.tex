\section{Описание методов отладки}
Для проверки корректности работы программы и отыскания утечек памяти я использовал утилиту valgrind.

Если необхоимо было выполнить один тест, то команда выглядела следующим образом:

\begin{alltt}
	(py37) ~ /DA/lab2 valgrind --leak-check=full --show-leak-kinds=all --track-origins=yes
	--verbose --log-file=valgrind-out.txt ./lab2ex <test2 \&\& tail -n5 valgrind-out.txt
\end{alltt}

Уже в ходе отладкки я, осознав небходимость в собственном генераторе тестов, создал таковой, и лишь благодарая ему смог отыскать и исправить все (регистрируемые чекером) ошибки в программе.

Созданный для проверки работы прогрмаммы генератор тестов работает следующим образом:

На первом этапе создется три строки со случайными именами файлов, в которые будет производится схранения дерева.

На втором - выполняется генерация первого тестового файла. В нем 50% команд
доступа являются командами добавления, остальные – команды поиска и удаления.
Таким образом, размер дерева наращивается. Все слова являются простыи комбинациями из 2-3 символов, чтобы обеспечить частые повторения. Периодически в последовательность команд добавляются операции загрузки и сохранения дерева в один
из трех файлов.

На третьем этапе генерируется схожий тестовый файл, однако распределение команд доступа в нем иное: 75% у команд удаления, 25 – у команд поска. Добавлений
не происходит. Таким образом при достаточной длине файла размер дерева медленно
уменьшается до 0. При этом не перестают выполняться периодические сохранения и
загрузки.

Для работы с этим тестером я создал дополнительный bash-скрипт, который выполняет генерацию 30 тестов и проверку коррестности работы прогрммы. В этой реализации генератор состоит из двух программ, каждая из которых генерирует один
файл с тестами.
\begin{lstlisting}[language=bash]
#!/bin/bash
for ((i=1;i<= 30;i++))
do
./gen.out && ./gen-2.out && valgrind --leak-check=full --show-leak-kinds=all --track-origins=yes --verbose --log-file=valgrind-out.txt ./lab2ex < test_of_absolutia > res-test&& tail -n5 valgrind-out.txt && valgrind --leak-check=full --show-leak-kinds=all --track-origins=yes --verbose --log-file=valgrind-out.txt ./lab2ex < second_test_of_absolutia > res-test && tail -n5 valgrind-out.txt 
done
\end{lstlisting}
\pagebreak

\section{Дневник разработки}
Дневник содержит в себе хронологию основных действий по отладке программы и список найденных ошибок

\begin{longtable}{|p{5cm}|p{5cm}|p{5cm}|}
	\hline
	\rowcolor{lightgray}
	\multicolumn{3}{|c|} {Дневник разработки}\\\hline
	Дата&Действие&Коментарий\\\hline
	20.01.2021&Реализаяция и отладка основных компонентов программы&В этот промежуток времени программа еще не работоспособна, производится содание классов дерева и узла, разработка парсера команд.\\\hline
	23.01.2021&Предварительная отладка&Создается простенькая программа, конвертирующая английйский текст в последовательность команд (на добавление, удаление и поиск слов). Ошибок не найдено.\\\hline
	23.01.2021&Проверка на предельных значениях&Производится ручная проверка на предельных длинах слов и значениях поля value. Ошибок не найдено.\\\hline
	25.01.2021&Первый залив на чекер&Наблюдается ошибка выполнения в тесте 4\\\hline
	28.01.2021&Обнаружена ошибка в алгоритме поиска узла в дереве&Алгоритм поиска не учитывает возможность того что дерево пусто. После исправления программа начинает проходить тест 4 почти до конца, 
	однако в итоге все равно ломается.\\\hline
	01.02.2021&Выполнена существенная оптимизация&В ходе отладки я решил заменить используемый до этого контейнер std::string на char*, отчего скорость работы программы выпросла почти вдвое.\\\hline
	02.02.2021&Выполнена существенная оптимизация&Я заменил инструмент вывода на экран с std::cout на printf, отчего скорость работы программы увеличилась ~ в полтора раза.\\\hline
	07.02.2021&Создание своего генератора тестов&Ввиду того что все предыдущие попытки отыскания ошибок не увенчались успехом, принимаю решение создать собственный генератор тестов, который сможет смоделировать большую часть допустимых последовательностей ввода.\\\hline
	08.02.2021&Обнаружена ошибка в алгоритме балансировки после удаления элемента&В ходе работы со сгенерированными тестами обнаружено, что несмотря на то что программа всегда корректно завершается, приблизительно в 30%
	случаев valgrind регистрирует обширную утечку памяти, по общему виду напоминающую утерю целой ветви дерева. При последующем анализе программы было обнаружено, что
	в определенных ситуациях при выполнении перебалансировки после удаления элемента не срабатывает рекурсивный вызов из-за чего функция возвращает в качестве нового корня элемент в центре дерева, а остальная
	часть структуры теряется. Ошибка исправлена. После исправления «ошибка выполнения на тесте 4» сменилась «на неправильный ответ на тесте 4».\\\hline
	08.02.2021&Обнаружена ошибка в алгоритме поиска узла в дереве&Так как для проверки равенства строк использовалась собственная функция strequal(), не учитывающая регистр букв в строках, а для,
	непосредственно, сравнения использовалась функция strcmp(), учитывающая регистр, порой strcmp() пускала поиск по неверному пути, отчего результат не соответствовал истине. Для обнаружения ошибки
	не потребовалось использование каких-либо специальных средств, Она была найдена в ходе мысленного анализа результатов выполнения тестов чекера. Ошибка исправлена путем переписывания strequal()
	таким образом, чтобы она покрывала функционал strcmp(). После исправления иных ошибок чекером обнаружено не было.\\\hline
\end{longtable}

\pagebreak
