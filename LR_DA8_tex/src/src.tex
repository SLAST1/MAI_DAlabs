\section{Описание}
Данный жадный алгоритм основан на расположении длин сторон в порядке убывания и проверке начиная сверху, берется три самых больших стороны считается
площадь, если такой треугольник возможен, делаем проверку сравнивая с текущей
наибольшей площадью, если значение больше, то запоминаем. Из-за сортировки требует $O(n log_n)$ времени.

\pagebreak

\section{Исходный код}

Код: main.cpp
\begin{lstlisting}[language=C++]
#include <iostream>
#include <vector>
#include <algorithm>
#include <cmath>
#include <iomanip>

bool CompareFunc(int const& lhs, int const& rhs) { return lhs > rhs; }

bool ValidTriangle(int const& s_1, int const& s_2, int const& s_3)
{
if((s_1 < (s_2 + s_3)) && (s_2 < (s_1 + s_3)) && (s_3 < (s_1 + s_2)))
return true;
else
return false;
}

double Area(int const& s_1, int const& s_2, int const& s_3)
{
double p = 0.5 * (s_1 + s_2 + s_3);
return sqrt(p) * sqrt(p - s_1) * sqrt(p - s_2) * sqrt(p - s_3);
}

int main()
{
std::vector<int> data;
int n = 0, s = 0, s_1 = 0, s_2 = 0, s_3 = 0;
double max_area = 0.0, cur_area = 0.0;

std::cin >> n;
for (int i = 0; i < n; ++i)
{
std::cin >> s;
data.push_back(s);
}

std::sort(data.begin(), data.end(), CompareFunc);

for(int i = 1; i < int(data.size() - 1); ++i)
{
if(data.size() < 3)
break;
if(ValidTriangle(data[i - 1], data[i], data[i + 1]))
{
cur_area = Area(data[i - 1], data[i], data[i + 1]);
if(cur_area > max_area)
{
max_area = cur_area;
s_1 = data[i + 1];
s_2 = data[i];
s_3 = data[i - 1];
}
}
}

if(max_area == 0)
std::cout << 0 << '\n';
else
{
printf("%.3f\n", max_area);
std::cout << s_1 << ' ' << s_2 << ' ' << s_3 << '\n';
}
return 0;
}
\end{lstlisting}

\pagebreak

\section{Консоль}

\begin{alltt}
(py37) ~ /DA_labs/lab8$ make
g++ -g -O2 -pedantic -std=c++17 -Wall -Wextra -Werror main.cpp -o solution
(py37) ~ /DA_labs/lab8$ cat test.txt
4
1
2
3
5
(py37) ~ /DA_labs/lab8$ ./solution <test.txt
0
\end{alltt}

\pagebreak

