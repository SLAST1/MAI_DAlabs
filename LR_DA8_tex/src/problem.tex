\CWHeader{Лабораторная работа \textnumero 8}

\CWProblem{
Разработать жадный алгоритм решения задачи, определяемой своим вариантом. Доказать его корректность, оценить скорость и объём затрачиваемой оперативной памяти.

Реализовать программу на языке С или С++, соответсвующую построенному алгоритму. Формат входных и выходных данных описан в варианте задания.

Заданы длины N отрезков, необходимо выбрать три таких отрезка, которые образовывали бы треугольник с максимальной площадью.

{\bfseries Формат входных данных:} На первой строке находится число N, за которым следует N строк с целыми числами-длинами отрезков.

{\bfseries Формат результата:} Если никакого треугольника из заданных отрезков составить
нельзя — 0, в противном случае на первой строке площадь треугольника с тремя
знаками после запятой, на второй строке — длины трёх отрезков, составляющих
этот треугольник. Длины должны быть отсортированы.
}
\pagebreak

