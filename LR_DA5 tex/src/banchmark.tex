\section{Тест производительности}

Тест состоит из 400 тыс. поисков паттерна в тексте для моего поиска с использованием суффиксного массива и для стандартного метода строк find, который модифицирован для поиска нескольких образцов в тексте. 
Паттерны повторяются по 4 штуки
- p1: «abc», p2: «de», p3: «ghhe», p4: «tcehj», p5: «abc» и т.д.

Текст: «abcgfhdeghheababctcehjtceghjtcehjdadeabcghheghhhheghhdeabcacbabchhjdetcehj»

Моя реализация:
\begin{alltt}
(py37) ~ /DA_labs/lab5$ make
g++ -g -O2 -pedantic -std=c++17 -Wall -Wextra -Werror main.cpp -o solution
(py37) ~ /DA_labs/lab5$ make bench
g++ -g -O2 -pedantic -std=c++17 -Wall -Wextra -Werror benchmark.cpp -o benchmark
(py37) ~ /DA_labs/lab5$ ./benchmark
Time for create suffix tree and suffix array: 0.0007106 seconds
Time for my find: 6.08816 seconds
Time for standart find: 1.66517 seconds
\end{alltt}

Как можно увидеть, время работы на построение суффиксново дерева и массива
крайне мало, а поиск в несколько раз медленнее, но если например нам важно сэкономить память, но не так критично время, то поиск с использованием суффиксного
массива вполне подходит.
\pagebreak

