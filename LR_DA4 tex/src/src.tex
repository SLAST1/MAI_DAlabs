\section{Описание}
Требуется написать реализацию алгоритма поиска образца, основанного на использовании Z-функции. 
Согласно [1], для строки S и позиции $i<1$ Z-функция $Z_i$(S) определяется как длина наибольшей подстроки S, которая начинается в $i$ и совпадает с префиксом S.

Для эффективного вычисления Z-функции необходимо ввести следующие понятия:
\begin{enumerate}
	\item Z-блок – для любой позиции $i > 1$, в которой $Z_i(S) > 0$, это интервал, начинающийся в $i$ и заканчивающийся в позиции $i + Z_i(S) - 1$.
	\item $r_i$ – крайняя правая позиция конца Z-блоков, начинающихся не позднее позиции $i$.
	\item $l_i$ – крайняя левая позиция Z-блока, начинающегося не позднее $i$, с наибольшей крайней правой позицией.
\end{enumerate}

Алгоритм Z-функции вычисляет значения $Z_i$, $r_i$ , $l_i$ последовательно для каждой позиции, начиная с $i = 2$. Для ускорения этого расчета используются уже вычисленные значения Z.
Пусть при работе алгоритма были вычислены Zi для $1 < i \leq k-1$ и текущие значения l и r. Тогда $Z_k$ и изменения для l и r можно определить следующим образом:
\begin{enumerate}
	\item Если $k > r$, то значение $Z_k$ вычисляется непосредственным сравнением подстрок, начиная с позиции $k$ и $с$ позиции 1. Длина совпадающей части и является $Z_k$, при $Z_k > 0$ также 
	изменяется значения $l = k$ и $r = k + Z_k - 1$.
	\item Если $k \leq r$, то позиция k содержится в текущем Z-блоке и подстрока $S[l..r]$
	совпадает с префиксом $S[1..r-l+ 1]$, а значит и символ на позиции k совпадает
	с символом в позиции $k = k - l + 1$. По тем же причинам подстрока $S[k..r]$
	(назовём её $\beta$) должная совпадать с подстрокой $S[k..Z_l]$, где ранее уже были
	вычислены значения Z-функции. Здесь возникает два возможных случая:
	Если $Z_k < | \beta |$, то $Z_k = Z_k$ и $r$, $l$ не изменяются.
	Если $Z_k \geq | \beta |$, то вся подстрока $S[k..r]$ должна быть префиксом S, однако не гарантируется то, что эта подстрока наибольшая совпадающая.
	Необходимо сравнить до несовпадения символы, начиная с позиции $r + 1$,
	с символами, начиная с позиции $| \beta |+ 1$. Пусть несовпадение произошло на
	символе $q \geq r + 1$. Тогда $Z_k$ полагается равным $q - k$, $r = q - 1$ и $l = k$.
\end{enumerate}	
\pagebreak

\section{Исходный код}
В рамках данной задачи работа алгоритма должна происходить не с буквами, а
целыми словами текста. Для корректного выполнения Z-функции нужно написать
функцию Equals(2), проверяющую два слова на равенства, с учетом регистронезависимости. Также определим метод CalculateZFunc(2), вычисляющий Z-функцию по
входной строке и сохраняющий её значения во входной массив. В свою очередь нахождение вхождений в тексте указанного образца будет осуществляться в другом методе – FindOccurs(3), использующем Z-функцию образца и вычисляющем значения
Z-функции для строки S = P\textdollar T, начиная с начала текста. Поскольку все значения этой Z-функции будут гарантированно меньше длины образца P (т.к. нельзя при сравнении уйти за символ \textdollar), 
то все значения в полученных Z-блоках могут быть найдены с использованием только значений Z-функции образца. Такой подход является довольно эффективным с точки зрения затрат памяти, поскольку не нужно
создавать массив с целыми числами Z-функции для целого текста. При нахождении
образца в тексте необходимо также вывести номер строки и номер слова, с которого начинается вхождение, для этого можно создать соответствующую структуру
TAnswer. Ниже приведён исходный код search.hpp с описанными методами:

Код: main.cpp
\begin{lstlisting}[language=C++]
#include "search.hpp"
int main() {
std::vector<int> zFunc;
std::vector<std::string> sample;
std::vector<int> stringEnd;
std::vector<std::string> text;
std::string word;
word.append(MAX_WORD_LENGTH, 0);
char c = getchar();
bool sampleFinish = false;
bool inputWord = false;
int index = 0;
while (c != EOF) {
if (c == '\n') {
if (inputWord) {
if (!sampleFinish) {
sample.push_back(word);
}
else {
text.push_back(word);
}   
for (int i = 0; i < index; ++i) {
word[i] = 0;
}
index = 0;
inputWord = false;
}
if (!sampleFinish) {
sampleFinish = true;
}
else {
stringEnd.push_back(text.size());
}
}
else if (c == '\t' || c == ' ') {
if (inputWord) {
if (!sampleFinish) {
sample.push_back(word);
}
else {
text.push_back(word);
}
for (int i = 0; i < index; ++i) {
word[i] = 0;
}
index = 0;
inputWord = false;
}
}
else {
word[index] = c;
++index;
inputWord = true;
}
c = getchar();
}
zFunc.assign(sample.size(), 0);
CalculateZFunc(sample, zFunc);
std::vector<TAnswer> ans = FindOccurs(sample, zFunc, text, stringEnd);
for (int j = 0; j < ans.size(); ++j) {
std::cout << ans[j].strPos << ", " << ans[j].wordPos << std::endl;
}
return 0;
}
\end{lstlisting}

В исходном коде main.cpp описан основной ход работы программы – чтение входного
образца и текста с учетом возможного произвольного числа пробелов между словами,
вызов функций вычисления Z-функции и поиска образца, а также вывод полученного
ответа

Код: search.hpp
\begin{lstlisting}[language=C++]
#include <iostream>
#include <vector>
#include <string>
#include <algorithm>
struct TAnswer {
int strPos;
int wordPos;
};
const int MAX_WORD_LENGTH = 16;
bool Equals(const std::string& first, const std::string& second) {
for (int i = 0; i < MAX_WORD_LENGTH; ++i) {
char c1 = ('A' <= first[i] && first[i] <= 'Z' ? first[i] - 'A' + 'a' : first[i]);
char c2 = ('A' <= second[i] && second[i] <= 'Z' ? second[i] - 'A' + 'a' : second[i]);
if (c1 != c2) {
return false;
}
}
return true;
}
void CalculateZFunc(const std::vector<std::string>& sample, std::vector<int>& zFunc) {
for (int i = 1, l = 0, r = 0; i < zFunc.size(); ++i) {
if (i <= r) {
zFunc[i] = std::min(zFunc[i - l], r - i + 1);
}
while (zFunc[i] + i < zFunc.size() && Equals(sample[zFunc[i]], sample[zFunc[i] + i])) {
++zFunc[i];
}
if (r < i + zFunc[i] - 1) {
l = i;
r = i + zFunc[i] - 1;
}
}
}
std::vector<TAnswer> FindOccurs(const std::vector<std::string>& sample, const std::vector<int>& zFunc, const std::vector<std::string>& text, const std::vector<int>& stringEnd) {
int value;
int wordPos = 0;
int strPos = 0;
std::vector<TAnswer> answer;
for (int i = 0, l = 0, r = 0; i < text.size(); ++i) {
value = 0;
while (i == stringEnd[strPos]) {
++strPos;
wordPos = 0;
}
++wordPos;
if (i <= r) {
value = std::min(zFunc[i - l], r - i + 1);
}
while (value < sample.size() && value + i < text.size() && Equals(sample[value], text[value + i])) {
++value;
}
if (r < i + value - 1) {
l = i;
r = i + value - 1;
}
if (value == sample.size()) {
answer.push_back({strPos + 1, wordPos});
}
}
return answer;
}
\end{lstlisting}

\pagebreak

\section{Консоль}

\begin{alltt}
(py37) ~ /DA_labs/lab4$ make
g++ -std=c++11 -O2 -Wextra -Wall -Werror -Wno-sign-compare -Wno-unused-result
-pedantic -o solution main.cpp
(py37) ~ /DA_labs/lab4$ cat test1.txt
cat dog cat dog bird
CAT dog CaT Dog Cat DOG bird CAT
dog cat dog bird
(py37) ~ /DA_labs/lab4$ ./solution <test1.txt
1,3
1,8
(py37) ~ /DA_labs/lab4$ cat test2.txt
mouse
mo use horse
cat dog cat mo
use mouse mouse
(py37) ~ /DA_labs/lab4$ ./solution <test2.txt
3,2
3,3
\end{alltt}

\pagebreak

